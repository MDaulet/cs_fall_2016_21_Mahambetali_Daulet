\documentclass{book} 

%% Language and font encodings 
\usepackage[english]{babel} 
\usepackage[utf8x]{inputenc} 
\usepackage[T1]{fontenc} 
\usepackage{fancyhdr} 
\usepackage[a4paper,top=3cm,bottom=2cm,left=4cm,right=2cm,marginparwidth=1.95cm]{geometry} 
\usepackage{xcolor}  
\usepackage{amsmath} 
\usepackage{graphicx} 
\usepackage[colorinlistoftodos]{todonotes} 
\usepackage[colorlinks=true, allcolors=blue]{hyperref} 
\usepackage{setspace} 

\begin{document} 
\pagestyle{fancy}  
\renewcommand{\headrulewidth}{0pt} 
\fancyhf{} % убираем текущие установки для колонтитулов 

\Huge\textbf{5}

\Huge\textbf{Taking out the big part}\\

\colorbox{lightgray}{
\begin{minipage}{\textwidth}
\large\textrm{5.1  Multiplication using one and few}

{5.2  Fractional changes and low-entropy expressions}

{5.3  Fractional changes with general exponents}

{5.4  Successive approximation: How deep is the well?}

{5.5  Daunting trigonometric integral}

{5.6 \textit{Summary and further problems}}
\end{minipage}}\\

\Large\textrm{In almost every quantitative problem,  the analysis simplifies when you
follow the proverbial advice of doing first things first.  First approximate
and understand the most important effect—the big part—then refine your
analysis and understanding.  This procedure of successive approximation
or “taking out the big part” generates meaningful, memorable, and usable
expressions.  The following examples introduce the related idea of low-
entropy expressions (Section 5.2) and analyze mental multiplication (Section 5.1),  exponentiation (Section 5.3),  quadratic equations (Section 5.4),
and a difficult trigonometric integral (Section 5.5).}
\\ 

\Large\textbf{5.1  Multiplication using one and few}

{The first illustration is a method of mental multiplication suited to rough,
back-of-the-envelope estimates.  The particular calculation is the storage
capacity of a data CD-ROM. A data CD-ROM has the same format and
storage capacity as a music CD, whose capacity can be estimated as the
product of three factors:}\\

\[\underbrace{1hr\times\frac{3600s}{1hr}}_{playing time}\times\underbrace{\frac{4.4\times10^{4}samples}{1s}}_{sample rate}\times{2 channels}\times\underbrace{\frac{16 bits}{1 sample}}_{sample size}\eqno{(5.1)}\]




%****************************************************************************************** 
%****************************************************************************************** 
\newpage 
\pagestyle{fancy} 
% этим мы убеждаемся, что заголовки глав и 
% разделов используют нижний регистр. 
\renewcommand{\headrulewidth}{0pt} 
\fancyhf{} % убираем текущие установки для колонтитулов 

\fancyhead[LE]{\large \textsl {\textbf{78}}} 
\fancyhead[RE]{\large \textsl{5 Taking out the big part}} 

\large\textrm{(In the sample-size factor, the two channels are for stereophonic sound.)}
\\ 
 
\colorbox{lightgray}{
\begin{minipage}{\textwidth}
\large\textbf{Problem 5.31  Sample rate} \\ 
\textmd{Look  up  the  Shannon–Nyquist  sampling  theorem  [22],  and  explain  why  the sample rate (the rate at which the sound pressure is measured) is roughly
40kHz.}\\

\large\textbf{Problem  5.2 Bits per sample} \\
\large\textmd{Because $2^{16}\sim10^{5}$,a 16-bit  sample—as  chosen  for  the  CD  format—requires electronics  accurate  to  roughly 0.001\%.  Why  didn’t  the  designers  of  the  CD format choose a much larger sample size, say 32 bits (per channel)?}\\

\large\textbf{Problem 5.3 Checking units}
\textmd{Check that all the units in the estimate divide out—except for the desired units of bits.}
\end{minipage}}\\

{Back-of-the-envelope calculations use rough estimates such as the playing
time and neglect important factors such as the bits devoted to error detec-
tion and correction.  In this and many other estimates, multiplication with
3 decimal places of accuracy would be overkill.  An approximate analysis
needs an approximate method of calculation.}\\

$\triangleright\large\textsl{What is the data capacity to within a factor of 2 ?}\\$

{The units (the biggest part!)  are bits (Problem 5.3), and the three numeri-cal factors contribute {3600$\times4.4\times10^{4}\times32$.  To estimate the product, split it into a big part and a correction.}\\
\textsl{The  big  part:} 
{The  most  important  factor  in  a  back-of-the-envelope  product usually comes from the powers of 10, so evaluate this big part first: 3600 contributes  three  powers  of 10,4.4$\times10^{4}$ contributes  four,  and 32 contributes one.  The eight powers of 10 produce a factor of $10^{8}$.}\\
\textsl{The correction:}
{After taking out the big part, the remaining part is a correction factor of 3.6$\times4.4\times3.2$ .  This product too is simplified by taking out its big part.  Round each factor to the closest number among three choices: 1,few,or 10. The invented number few lies midway between 1 and 10:It is the geometric mean of 1 and 10,so (few)2=10 and few$\approx3$.In the product 3.6$\times4.4\times3.2$, each factor rounds to few, so 3.6$\times4.4\times3.2\approx(few)^{3}$ or roughly 30}\\

{The units, the powers of 10, and the correction factor combine to give}\\

\[capacity\sim10^{8}\times30 bits=3\times10^{9} bits.\eqno{(5.2)}\]

%************************************************************************************* 
%*********************************************************************************** 
\newpage 
\pagestyle{fancy} 
% этим мы убеждаемся, что заголовки глав и 
% разделов используют нижний регистр. 
\renewcommand{\headrulewidth}{0pt} 
\fancyhf{} % убираем текущие установки для колонтитулов 

\fancyhead[RO]{\large \textsl {\textbf{79}}} 
\fancyhead[LO]{\large \textsl{5.2 Fractional changes and low-entropy expressions}} 
\Large\textrm{This estimate is within a factor of 2 of the exact product (Problem 5.4), which is itself close to the actual capacity of 5.6$\times10^{9}$
bits.}\\ 

\colorbox{lightgray}{
\begin{minipage}{\textwidth}
\large\textbf{Problem  5.4 Underestimate or overestimate?}\\
\large\textmd{Does 3$\times10^{9}$ overestimate or underestimate 3600$\times4.4\times10^{4}\times32$? Check your reasoning by computing the exact product.}\\

\large\textbf{Problem  5.5  More practice}\\
\large\textmd{Use the one-or-few method of multiplication to perform the following calculations mentally; then compare the approximate and actual products.}\\

\large\textmd{a.  161$\times294\times280\times$438.  The actual product is roughly 5.8$\times10^{9}$.}\\

\large\textmd{b.  Earth’s surface area A=4${\pi}$$R^{2}$, where the radius is R$\sim6\times10^{6}$m.  The actual surface area is roughly 5.1$\times10^{14}m^{2}$.}
\end{minipage}}\\

\Large\textbf{5.2 Fractional changes and low-entropy expressions}\\

{Using the one-or-few method for mental multiplication is fast.  For example, 3.15$\times7.21$ quickly becomes few$\times10^{1}\sim$30 , which is within 50\% of the exact product 22.7115 .  To get a more accurate estimate, round 3.15 to 3 and 7.21 to 7 . Their product 21 is in error by only 8\%.  To reduce the error further, one could split 3.15$\times7.21$ into a big part and an additive correction.  This decomposition produces}
\[{(3+0.15)(7+0.21)=}\underbrace{3\times7}_{big part}+\underbrace{0.15\times7+3\times0.21+0.15\times0.21}_{additive correction}{.}\eqno{(5.3)}\]\\
{The approach is sound, but the literal application of taking out the big part  produces  a  messy  correction  that  is  hard  to  remember  and  understand.   Slightly  modified,  however,  taking  out  the  big  part  provides  a clean  and  intuitive  correction.   As  gravy,  developing  the  improved  correction introduces two important street-fighting ideas:  fractional changes (Section 5.2.1) and low-entropy expressions (Section 5.2.2).  The improved correction will then, as a first of many uses, help us estimate the energy saved by highway speed limits (Section 5.2.3).}\\

\large\textbf{5.2.1 Fractional changes}\\
{The hygienic alternative to an additive correction is to split the product into a big part and a \textit{multiplicative} correction:}

%************************************************************************************* 
%*********************************************************************************** 
\newpage 
\pagestyle{fancy} 
% этим мы убеждаемся, что заголовки глав и 
% разделов используют нижний регистр. 
\renewcommand{\headrulewidth}{0pt} 
\fancyhf{} % убираем текущие установки для колонтитулов 

\fancyhead[LE]{\large \textsl {\textbf{80}}} 
\fancyhead[RE]{\large \textsl{5 Taking out the big part}} 

\[{3.15\times7.21=}\underbrace{3\times7}_{big part}\times\underbrace{(1+0.05)\times(1+0.03)}_{correction factor}\eqno{(5.3)}\]\\
$\triangleright\large\textsl{Can you find a picture for the correction factor?}\\$

\large\textrm{The correction factor is the area of a rectangle with}\\ 
{width 1+0.05 and height 1+0.03.  The rectangle}\\ 
{contains one subrectangle for each term in the ex-}\\
{pansion of (1+0.05)$\times(1+0.03)$. Their combined}\\ 
{area of roughly 1+0.05+0.03 represents an 8 \%}\\ 
{fractional increase over the big part. The big part}\\ 
{is 21 , and 8 \%of it is 1.68 ,so 3.15$\times7.21$=22.68 ,}\\ 
{which is within 0.14 \% of the exact product.}\\

\colorbox{lightgray}{
\begin{minipage}{\textwidth}
\large\textbf{Problem 5.6 Picture for the fractional error} \\

{What is the pictorial explanation for the fractional error of roughly 0.15\%?}\\

\large\textbf{Problem  5.7 Try it yourself}\\

{Estimate 245$\times$42 by rounding each factor to a nearby multiple of 10, and compare this big part with the exact product.  Then draw a rectangle for the correction factor, estimate its area, and correct the big part.}
\end{minipage}}\\

\large\textbf{5.2.2  Low-entropy expressions}\\

{The correction to 3.15$\times7.21$ was complicated as an absolute or additive change but simple as a fractional change.  This contrast is general.  Using the additive correction, a two-factor product becomes}\\

{(x+$\Delta x$)(y+$\Delta y$)=xy+$\underbrace{y\Delta x+x\Delta y+\Delta x\Delta y}_{additive correction}$}\\

\colorbox{lightgray}{
\begin{minipage}{\textwidth}
\large\textbf{Problem  5.8 Rectangle picture}\\

{Draw a rectangle representing the expansion}
\[{(x+\Delta x)(y+\Delta y)=xy+y\Delta x+x\Delta y+\Delta x\Delta y}\eqno{(5.6)}\]\\
\end{minipage}}\\

{When the absolute changes $\Delta x$ and $\Delta y$ are small (x$\ll\Delta x\,and\, y\ll\Delta y$),the correction simplifies to x$\Delta y$+y$\Delta x$ , but even so it is hard to remember because it has many plausible but incorrect alternatives.  For example, it could  plausibly  contain  terms  such  as $\Delta x\Delta y$ x$\Delta x$ ,or y$\Delta y$.   The  extent}
\end{document}
